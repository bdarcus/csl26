% csln-example.tex — Full demonstration of the CSLN LuaLaTeX package
%
% Compile with:
%   lualatex --shell-escape csln-example.tex
%
% Note: --shell-escape is required because the CSLN package uses LuaTeX's
% FFI to load a native shared library.  Without it, ffi.load is blocked.
%
% Prerequisites:
%   1. Build the shared library:
%        cargo build --package csln_processor --release --features ffi
%   2. Either set CSLN_LIB_PATH to the full path of libcsln_processor.dylib/.so,
%      or copy/symlink it into this directory.
%   3. Either set CSLN_LUA_PATH to the full path of csln.lua,
%      or place csln.lua and csln.sty in the same directory as this file.
%
% The style path below resolves relative to this file; adjust if needed.

\documentclass[12pt, a4paper]{article}

% ---------------------------------------------------------------------------
% CSLN citation package
%   style   = path to a CSLN YAML style file (extension optional)
%   bibfile = path to a CSLN YAML bibliography (extension optional)
%             Use bibfile=refs.bib for a biblatex .bib file instead.
% ---------------------------------------------------------------------------
\usepackage[
  style   = ../../styles/apa-7th,
  bibfile = example-refs
]{csln}

% Standard LaTeX preamble
\usepackage{fontspec}              % LuaLaTeX font support
\usepackage[hidelinks]{hyperref}   % clickable DOIs / URLs without coloured boxes
\usepackage{microtype}             % better typography
\usepackage{geometry}
\geometry{margin=2.5cm}

\setmainfont{Latin Modern Roman}

\title{Paradigm Shifts and Machine Learning:\\
       A Demonstration of CSLN in Lua\LaTeX}
\author{Example Author}
\date{2026}

\begin{document}

\maketitle

\begin{abstract}
This document demonstrates the CSLN (\textit{Citation Style Language Next})
citation package for Lua\LaTeX.
Citations are rendered live by the CSLN Rust processor via a LuaJIT~FFI
call --- no Biber invocation, no \texttt{.bbl} file, and no shell-escape
are required.
The bibliography at the end of this document is likewise produced entirely
within a single \texttt{lualatex} pass.
\end{abstract}

% ---------------------------------------------------------------------------
\section{Introduction}
% ---------------------------------------------------------------------------

The history of science does not progress by the simple accumulation of
facts.
\textcite{kuhn1962} argued instead that mature scientific disciplines are
organised around \emph{paradigms} --- shared frameworks of concepts,
methods, and exemplars that define what counts as a legitimate problem and
a legitimate solution.
Normal science, in this account, is puzzle-solving within a paradigm
\cite{kuhn1962}.

% ---------------------------------------------------------------------------
\section{Neural Networks as a Paradigm Shift}
% ---------------------------------------------------------------------------

Modern deep learning constitutes, by Kuhn's criteria, a paradigm shift
in how we approach pattern recognition and inference
\cite{lecun2015}.
The transformer architecture \cite{vaswani2017} has since displaced
recurrent models across virtually every natural-language processing task,
and Hawking's popular account of physics \cite{hawking1988} provides a
useful counterpoint illustrating how conceptual frameworks reach lay
audiences.

\subsection{Multi-cite and locator examples}

Multiple works can be cited together in a single parenthetical:
\cites{kuhn1962, lecun2015, vaswani2017}

Pinpoint citations with locators also work.
The canonical argument for deliberate practice appears in
\textcite[p.~690]{ericsson2006}, and the statistical details of deep
learning are spelled out at \cite[pp.~438--440]{lecun2015}.

Affixes on individual citation items use the \cs{citestart}/\cs{citeitem}/%
\cs{citeend} API for full control:
%
\citestart
  \citeitem[p.~5999]{vaswani2017}
  \citeitem[p.~437]{lecun2015}
\citeend

% ---------------------------------------------------------------------------
\section{Expertise and Deliberate Practice}
% ---------------------------------------------------------------------------

The Ericsson model of expert performance \cite{ericsson2006} holds that
domain mastery requires structured, goal-directed practice rather than
mere repetition.
Weinberg and Freedman's classic study of programmer behaviour
\cite{weinberg1971} reached analogous conclusions in the context of
software development.

For macro-level development trends, institutional sources play a similar
evidential role \cite{worldbank2023}.

% ---------------------------------------------------------------------------
\section{Conclusion}
% ---------------------------------------------------------------------------

The CSLN processor renders these citations --- non-integral parentheticals,
integral (\texttt{textcite}) forms, multi-key groups, and page-level
locators --- directly from Rust during the \LuaLaTeX{} run.
No external bibliography processor is required.

% ---------------------------------------------------------------------------
% Bibliography
% ---------------------------------------------------------------------------
\newpage
\section*{References}
\printcslnbibliography

\end{document}
